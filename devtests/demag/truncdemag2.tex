\documentclass[12pt,a4paper,notitlepage]{article}
\usepackage[utf8]{inputenc}
\input magstyle.sty

% Title Page
\title{Calculation of truncated Demagnetisation Field - Direct Approach}
\author{Gabriel Balaban and University of Southampton}

\begin{document}
\maketitle

\abstract{}
In this document the mathematics of the program truncdemag2.py are explained. The demagnetisation field is found
using an approach that truncates an infinite domain and uses two scalar potentials.
\end{abstract}
\newpage
\subsection*{The Demagnetisation Field inside a Magnetic Body} 
The demagnetisation field $\dmag$ related to a magnetic body is generated by its magnetisation field $\magn$ ,
 and tends to oppose it.In this context the region in space occupied by the magnetic object of interest 
(usually a ferromagnetic material) is denoted by  $\omM$.
The complement of this region $\omV$ is considered to be a vacuum, which extends to infinity and over which
the demagnetisation field
$\dmag $ is also defined.
According to basic electrodynamics the two fields are related by

\begin{equation}\label{dmagdef} \nabla \cdotp \dmag = - \nabla \cdotp \magn \  \mbox{in} \ \omM \end{equation}

\noindent The assumption is also made that there are no free currents and that the electric displacement field does not change over time so that
\[ \nabla \times \dmag = 0\]

\noindent This means that we can introduce a magnetic potential function $\phi$, such that

\[ \dmag = - \nabla \phi \]

\noindent which means that 

\[ - \nabla^2 \phi = \nabla \cdotp \magn \]

\noindent so that our demagnetisation field can be obtained by solving a Poisson problem.

\subsection*{The Fredkin-Koehler approach} 
The Fredkin-Koehler approach splits the magnetic potential $\phi$ into two parts $\phi_1$ and $\phi_2$, 
\[ \phi = \phi_1 + \phi_2 \]
\noindent in order to get $\phi_2$ using the hybrid boundary/finite element method(FEM/BEM).
 Here we solve for the two directly using domain truncation.
\\
\\
\noindent $\phi_1$ is defined to be the solution of the Neumann problem:

\[ 
\left\{
\begin{array}{lr}
- \nabla^2 \phi_1 = \nabla \cdotp \magn  & \mbox{in } \ \omM \\
 \frac{\partial \phi_1}{\partial n}  = -n \cdotp \magn & \mbox{on } \  \partial \omM \\
 \phi_1 = 0 & \mbox{on} \ \ \omV
\end{array}
\right. 
\]
\\
\\
\noindent and $\phi_2$ is defined to be the solution of the Laplace problem
\[ 
\left\{
\begin{array}{lr}
- \nabla^2 \phi_2 = 0 & \mbox{in } \ \omM \cup \omV \\
\mbox{jump}(\phi_2) = \phi_1 & \mbox{on} \ \domM \\
\ \phi_2 \rightarrow 0 & \ \mbox{as} \ \overrightarrow{r} \in \omV \rightarrow \infty
\end{array}
\right. 
\]
\\
\\
\noindent Adding together the two potentials $\phi_1$ and $\phi_2$ should result in the solution to the problem described in
truncdemag1. 

\subsection*{Truncation of the Domain}
Since a computer cannot simulate an infinite domain there needs to be some way of dealing with the open boundary condition 
\[ \ \phi_2 \rightarrow 0 & \ \mbox{as} \ \overrightarrow{r} \in \omV \rightarrow \infty \]
Here the domain \OmV is truncated, using the rule of thumb that it should be about 5 times as large as \OmM in order to get
decent results. 

\end{document}          
