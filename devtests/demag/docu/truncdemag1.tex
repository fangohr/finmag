\documentclass[12pt,a4paper,notitlepage]{article}
\usepackage[utf8]{inputenc}
\input magstyle.sty

% Title Page
\title{Calculation of truncated Demagnetisation Field - Direct Approach}
\author{Gabriel Balaban and University of Southampton}

\begin{document}
\maketitle

\abstract{}
In this document the mathematics of the program truncdemag1.py are explained. The demagnetisation field is found
using an approach that truncates an infinite domain and uses one scalar potential.
\end{abstract}
\newpage
\subsection*{The Demagnetisation Field inside a Magnetic Body} 
The demagnetisation field $\dmag$ related to a magnetic body is generated by its magnetisation field $\magn$ ,
 and tends to oppose it.In this context the region in space occupied by the magnetic object of interest 
(usually a ferromagnetic material) is denoted by  $\omM$.
The complement of this region $\omV$ is considered to be a vacuum, which extends to infinity and over which
the demagnetisation field
$\dmag $ is also defined.
According to basic electrodynamics the two fields are related by

\begin{equation}\label{dmagdef} \nabla \cdotp \dmag = - \nabla \cdotp \magn \  \mbox{in} \ \omM \end{equation}

\noindent The assumption is also made that there are no free currents and that the electric displacement field does not change over time so that
\begin{equation}\label{nocurl} \nabla \times \dmag = 0 \end{equation}

\noindent This means that we can introduce a magnetic potential function $\phi$, such that

\[ \dmag = - \nabla \phi \]

\noindent which means that 

\[ - \nabla^2 \phi = \nabla \cdotp \magn \]

\noindent so that our demagnetisation field can be obtained by solving a Poisson problem.

\subsection*{Poisson Problem for the Scalar Potential}
The poisson problem for the scalar potential $\phi$ reads
\[ 
\left\{
\begin{array}{lr}
- \nabla^2 \phi = \nabla \cdotp \magn  & \mbox{in } \ \omM \\
- \nabla^2 \phi = 0 & \mbox{in } \ \omV \\
& \\
\mbox{jump}( \frac{\partial \phi}{\partial n} ) = -n \cdotp \magn & \mbox{on } \  \domM \\
\  \lim_{\overrightarrow{r} \in \omM \rightarrow \domM} \phi =  \lim_{\overrightarrow{r} \in \omV \rightarrow \domM} \phi 
& \mbox{on } \domM \\
& \\
\ \phi \rightarrow 0 & \ \mbox{as} \ \overrightarrow{r} \in \omV \rightarrow \infty 
\end{array}
\right. 
\]
\noindent Where $n$ denotes the unit outward normal. The prescribed jump in the normal derivative is a 
consequence of the divergence theorem applied to the definition of $\dmag$, equation \ref{dmagdef}.
Using equation \ref{nocurl} and stokes theorem we see that the tangential component of $\nabla \phi$ or $\dmag$ is
continuous over $\domM$. Which means that the scalar potential only differs by a constant as it crosses $\domM$.
This constant is set to 0, which means that $\phi$ is continuous over $\domM$. Finally we expect the demag fiel $\dmag$ 
to decay to 0 far away from the magnetic body, which results in the open boundary condition
$| \dmag | \rightarrow 0 & \ \mbox{as} \ \overrightarrow{r} \in \omV \rightarrow \infty$ for the potential $\phi$. 

\subsection*{Truncation of the Domain}
Since a computer cannot simulate an infinite domain there needs to be some way of dealing with the open boundary condition 
\[ \ \phi \rightarrow 0 & \ \mbox{as} \ \overrightarrow{r} \in \omV \rightarrow \infty \]
Here the domain \OmV is truncated, using the rule of thumb that it should be about 5 times as large as \OmM in order to get
decent results. 

\end{document}          
